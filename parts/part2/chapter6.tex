\chapter{选择器}

\section{选一个标签——标签选择器}

\subsection{标签选择器}

标签选择器其实就是HTML代码中的标签。 \\

\begin{lstlisting}[style=htmlcssjs]
tag_selector {
    attribute: value;
}
\end{lstlisting}

\begin{lstlisting}[style=htmlcssjs, title=标签选择器]
<!DOCTYPE html>
<html lang="en">
<head>
    <meta charset="UTF-8">
    <title>标签选择器</title>
    <style type="text/css">
        h2 {
            color: green;
            font-size: 30px;
        }
    </style>
</head>
<body>
    <h2>CSS特点</h2>
    <p>CSS为HTML标记语言提供了一种样式描述。</p>
</body>
</html>
\end{lstlisting}

\newpage

\section{再选一个类——类选择器}

\subsection{类选择器}

类选择器在CSS样式中是最常用的。类选择器使用【.】开头,后加类选择器的名称,CSS样式代码会被作用到属于该类的HTML标签中。在标签中使用class属性为标签设置一个类。 \\

类选择器与标签是多对多的关系,即类选择器名称可以多个标签共用,一个元素可以用多个class,一个class值可以对应多个元素。多个class值之间使用空格分隔。

\begin{lstlisting}[style=htmlcssjs, title=类选择器]
<!DOCTYPE html>
<html lang="en">
<head>
    <meta charset="UTF-8">
    <title>类选择器</title>
    <style type="text/css">
        .title {
            color: green;
        }
    </style>
</head>
<body>
    <h3 class="title">丰富的样式定义</h3>
    <p>CSS提供了丰富的文档样式外观。</p>
    <h3 class="title">易于使用和修改</h3>
    <p>CSS样式表可以将所有的样式声明统一存放,进行统一管理。</p>
</body>
</html>
\end{lstlisting}

\newpage

\section{取个唯一表示——ID选择器}

\subsection{ID选择器}

使用ID选择器,必须给标签添加上id属性,即为标签设置id属性。ID选择器名称的前面使用【\#】。 \\

ID选择器与标签是一对一的关系,即一个元素只能有一个id值,一个id值只能对应一个元素。id是全局唯一的,就像身份证号码一样。

\begin{lstlisting}[style=htmlcssjs, title=ID选择器]
<!DOCTYPE html>
<html lang="en">
<head>
    <meta charset="UTF-8">
    <title>ID选择器</title>
    <style type="text/css">
        div {
            width: 100px;
            height: 100px;
        }

        #square1 {
            background-color: red;
        }
        #square2 {
            background-color: blue;
        }
    </style>
</head>
<body>
    <div id="square1"></div>
    <div id="square2"></div>
</body>
</html>
\end{lstlisting}

\newpage

\section{捡了个儿子——子选择器}

\subsection{子选择器}

子选择器【>】,用于选择指定标签元素的第一代子元素。

\begin{lstlisting}[style=htmlcssjs, title=子选择器]
<!DOCTYPE html>
<html lang="en">
<head>
    <meta charset="UTF-8">
    <title>子选择器</title>
    <style type="text/css">
        .food > li {
            border: 1px solid red;
        }
    </style>
</head>
<body>
    <h2>食物</h2>
    <ul class="food">
        <li>水果
            <ul>
                <li>苹果</li>
                <li>香蕉</li>
            </ul>
        </li>
        <li>蔬菜
            <ul>
                <li>白菜</li>
                <li>油菜</li>
            </ul>
        </li>
    </ul>
</body>
</html>
\end{lstlisting}

\newpage

\section{这么快就当爷爷了——后代选择器}

\subsection{后代选择器}

后代选择器,也称包含选择器,用于选择指定标签元素的后辈元素。 \\

\begin{lstlisting}[style=htmlcssjs]
ancestor_selector descendant_selector {
    attribute: value;
}
\end{lstlisting}

后代选择器与子选择器的区别在于,子选择器仅是指它的直接后代,而后代选择器是作用于所有子后代元素。

\begin{lstlisting}[style=htmlcssjs, title=后代选择器]
<!DOCTYPE html>
<html lang="en">
<head>
    <meta charset="UTF-8">
    <title>后代选择器</title>
    <style type="text/css">
        .food li {
            border: 1px solid red;
        }
    </style>
</head>
<body>
    <h2>食物</h2>
    <ul class="food">
        <li>
            水果
            <ul>
                <li>苹果</li>
                <li>香蕉</li>
                <li>橘子</li>
            </ul>
        </li>
        <li>
            蔬菜
            <ul>
                <li>白菜</li>
                <li>油菜</li>
                <li>卷心菜</li>
            </ul>
        </li>
    </ul>
</body>
</html>
\end{lstlisting}

\newpage

\section{我全都要——通配符选择器}

\subsection{通配符选择器}

通配符选择器【*】,也称通用选择器,是功能最强大的选择器,用于匹配HTML中所有的标签元素,包括\lstinline|<html>|、\lstinline|<body>|等。

\begin{lstlisting}[style=htmlcssjs, title=通配符选择器]
<!DOCTYPE html>
<html lang="en">
<head>
    <meta charset="UTF-8">
    <title>通配符选择器</title>
    <style type="text/css">
        * {
            background-color: yellow;
        }
    </style>
</head>
<body>

</body>
</html>
\end{lstlisting}

\newpage

\section{给选择器分个组——分组选择器}

\subsection{分组选择器}

分组选择器【,】,用于为HTML中多个标签元素设置同一个样式。

\begin{lstlisting}[style=htmlcssjs, title=分组选择器]
<!DOCTYPE html>
<html lang="en">
<head>
    <meta charset="UTF-8">
    <title>分组选择器</title>
    <style type="text/css">
        h1, h2, h3 {
            color: red;
        }
    </style>
</head>
<body>
    <h1>HTML</h1>
    <h2>CSS</h2>
    <h3>JavaScript</h3>
</body>
</html>
\end{lstlisting}

\newpage

\section{伪装者——伪类选择器}

\subsection{伪类选择器}

伪类选择器【:】允许给HTML标签的某种状态设置样式,例如给一个标签元素的鼠标滑过的状态设置字体颜色。

\begin{table}[H]
    \centering
    \setlength{\tabcolsep}{5mm}{
        \begin{tabular}{|c|c|}
            \hline
            \textbf{伪类选择器} & \textbf{功能}    \\
            \hline
            :link               & 未访问           \\
            \hline
            :visited            & 已访问           \\
            \hline
            :hover              & 鼠标悬停         \\
            \hline
            :active             & 鼠标按下         \\
            \hline
            :enabled            & 可用的时候触发   \\
            \hline
            :disabled           & 不可用的时候触发 \\
            \hline
        \end{tabular}
    }
    \caption{常用伪类选择器}
\end{table}

到目前为止,可以兼容所有浏览器的伪类选择器就是在\lstinline|<a>|上使用:hover。

\begin{lstlisting}[style=htmlcssjs, title=伪类选择器]
<!DOCTYPE html>
<html lang="en">
<head>
    <meta charset="UTF-8">
    <title>伪类选择器</title>
    <style type="text/css">
        a:hover {
            color: orange;
        }
    </style>
</head>
<body>
    <a href="http://www.baidu.com">百度一下,你就知道</a>
</body>
</html>
\end{lstlisting}

\newpage

\section{为所欲为——选择器最高层级!important}

\subsection{选择器优先级}

每个选择器都是有优先级的,如果一个元素使用了多个选择器,则会按照选择器的优先级来给定样式。 \\

选择器的优先级依次是:内联样式 > ID选择器 > 类选择器 > 标签选择器 > 通配符选择器。

\subsection{选择器权重}

浏览器是根据权值来判断使用哪种CSS样式的,权值高的优先级更高。

\begin{table}[H]
    \centering
    \setlength{\tabcolsep}{5mm}{
        \begin{tabular}{|c|c|}
            \hline
            \textbf{选择器} & \textbf{权值} \\
            \hline
            !important      & $ \infty $    \\
            \hline
            行间样式        & 1000          \\
            \hline
            ID              & 100           \\
            \hline
            class           & 10            \\
            \hline
            标签            & 1             \\
            \hline
            通配符          & 0             \\
            \hline
        \end{tabular}
    }
    \caption{选择器权重}
\end{table}

\subsection{!important}

有些特殊情况需要为某些样式设置具有最高权值,这时候可以使用\lstinline|!important|,注意\lstinline|!important|要写在分号的前面。

\begin{lstlisting}[style=htmlcssjs, title=!important]
<!DOCTYPE html>
<html lang="en">
<head>
    <meta charset="UTF-8">
    <title>!important</title>
    <style type="text/css">
        h1 {
            color: red !important;
        }

        h1 {
            color: blue;
        }
    </style>
</head>
<body>
    <h1>为所欲为</h1>
</body>
</html>
\end{lstlisting}

\newpage