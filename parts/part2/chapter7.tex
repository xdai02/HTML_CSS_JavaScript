\chapter{字体、文本样式}

\section{字体样式}

\subsection{字体样式}

使用CSS样式可以为网页中的文字设置字体。注意不要设置不常用的字体,因为如果用户本地电脑上没有安装该字体,就会显示浏览器默认的字体。 \\

浏览器默认的字号为16px,使用font-size可以修改字号大小。 \\

为文字设置粗体是有单独的CSS样式来实现的,再也不用为了实现粗体样式而使用\lstinline|<h1>|-\lstinline|<h6>|或\lstinline|<strong>|了。 \\

font-weight的默认值为normal,通过设置属性值为lighter、bold、bolder或100-900之间的整百数值改变文字的粗细。注意,字体能否被bolder或lighter更改取决于字体包是否存在该样式。 \\

font-style可以设置字体样式,并且有3种设置方式:

\begin{enumerate}
    \item 正常字体为normal,也是font-style的默认值
    \item italic为字体设置为斜体,用于字体本身就有倾斜的样式
    \item oblique强制将字体倾斜
\end{enumerate}

\begin{lstlisting}[style=htmlcssjs, title=字体样式]
<!DOCTYPE html>
<html lang="en">
<head>
    <meta charset="UTF-8">
    <title>字体样式</title>
    <style type="text/css">
        p {
            font-family: "arial";
            font-size: 20px;
            font-weight: bold;
            font-style: italic;
        }   
    </style>
</head>
<body>
    <p>Cascading Style Sheets (CSS) is a style sheet language used for describing the presentation of a document written in a markup language such as HTML.</p>
</body>
</html>
\end{lstlisting}

\newpage

\section{上个色——color}

\subsection{color}

color属性可以设置字体颜色。color的值有3种设置方式:

\begin{enumerate}
    \item 英文命令颜色。 \\
          \begin{lstlisting}[style=htmlcssjs]
color: red;
    \end{lstlisting}

    \item 十六进制颜色代码:使用6位十六进制数表示光学三原色“红绿蓝”。
          \begin{table}[H]
              \centering
              \setlength{\tabcolsep}{5mm}{
                  \begin{tabular}{|c|c|c|}
                      \hline
                      \textbf{R} & \textbf{H} & \textbf{B} \\
                      \hline
                      00 - FF    & 00 - FF    & 00 - FF    \\
                      \hline
                  \end{tabular}
              }
              \caption{颜色代码}
          \end{table}
          如果每两位十六进制数都相同,可简写。
          \begin{table}[H]
              \centering
              \setlength{\tabcolsep}{5mm}{
                  \begin{tabular}{|c|c|}
                      \hline
                      \textbf{颜色代码} & \textbf{颜色} \\
                      \hline
                      \#F00             & 红色          \\
                      \hline
                      \#0F0             & 绿色          \\
                      \hline
                      \#00F             & 蓝色          \\
                      \hline
                      \#000             & 黑色          \\
                      \hline
                      \#FFF             & 白色          \\
                      \hline
                      \#0FF             & 青色          \\
                      \hline
                      \#F40             & 淘宝红        \\
                      \hline
                  \end{tabular}
              }
              \caption{常见颜色代码}
          \end{table}

    \item 颜色函数rgb():由光学三原色RGB的比例来配色。rgb()函数中每一项的值可以是0-255之间的整数,也可以是0\%-100\%的百分数。 \\
          \begin{lstlisting}[style=htmlcssjs]
color: rgb(133, 45, 200);
color: rgb(20%, 33%, 25%);
    \end{lstlisting}
\end{enumerate}

\newpage

\section{文本样式}

\subsection{文本样式}

text-decoration可以设置添加到文本的修饰,默认值为none。属性值为underline为文本添加下划线,属性值为overline为文本添加上划线,属性值为line-through为添加穿过文本的线,一般用于商品折扣价。 \\

使用line-height可以设置段落中的行间距离(行高)。使用text-align可以为文本设置对齐方式,属性值包括left、right和center。

\begin{lstlisting}[style=htmlcssjs, title=文本样式]
<!DOCTYPE html>
<html lang="en">
<head>
    <meta charset="UTF-8">
    <title>文本样式</title>
    <style type="text/css">
        h2, p {
            text-decoration: underline;
            line-height: 2em;   /* 两倍行间距 */
            text-align: center;
        }
    </style>
</head>
<body>
    <h2>《望庐山瀑布》</h2>
    <p>唐·李白</p>
    <p>
        日照香炉生紫烟,<br/>
        遥看瀑布挂前川。<br/>
        飞流直下三千尺,<br/>
        疑是银河落九天。
    </p>
</body>
</html>
\end{lstlisting}

\newpage