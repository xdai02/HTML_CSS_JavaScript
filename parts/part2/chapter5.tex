\chapter{CSS简介}

\section{给HTML打扮——CSS样式}

\subsection{层叠样式表(CSS, Cascading Style Sheets)}

CSS主要用于定义HTML内容在浏览器内的显示样式,如文字大小、颜色、字体加粗等。使用CSS样式的一个好处是通过定义某个样式,可以让不同网页位置的问题有着统一的字体、字号或颜色等。CSS样式由选择器和声明组成,而声明又由属性和值组成。 \\

\begin{lstlisting}[style=htmlcssjs]
selector {
    attribute: value;
}
\end{lstlisting}

选择器也称选择符,用于指明网页中要应用样式规则的元素。声明的内容写在大括号【{}】内,属性和值之前用冒号【:】分隔。当有多条声明时,中间可以用分号【;】分隔。 最后一条声明可以没有分号,但是为了后续修改方便,一般也加上分号。为了使样式更加容易阅读,可以将每条声明单独成行。

\begin{lstlisting}[style=htmlcssjs, title=修改字体大小和颜色]
<!DOCTYPE html>
<html lang="en">
<head>
    <meta charset="UTF-8">
    <title>修改字体大小和颜色</title>
    <style type="text/css">
        p {
            font-size: 20px;
            color: red;
        }
    </style>
</head>
<body>
    <p>修改字体大小和颜色</p>
</body>
</html>
\end{lstlisting}

\newpage

\section{既然那么好,那就引入CSS吧——内联式CSS}

\subsection{内联式CSS}

内联式CSS样式,也称行间样式,就是把CSS代码直接写在现有的HTML标签中。注意CSS样式必须写在元素的开始标签里,不能在结束标签里。 \\

