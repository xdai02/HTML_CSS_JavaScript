\chapter{CSS简介}

\section{给HTML打扮——CSS样式}

\subsection{层叠样式表(CSS, Cascading Style Sheets)}

CSS主要用于定义HTML内容在浏览器内的显示样式,如文字大小、颜色、字体加粗等。使用CSS样式的一个好处是通过定义某个样式,可以让不同网页位置的问题有着统一的字体、字号或颜色等。CSS样式由选择器和声明组成,而声明又由属性和值组成。 \\

选择器也称选择符,用于指明网页中要应用样式规则的元素。声明的内容写在大括号【\{\}】内,属性和值之前用冒号【:】分隔。当有多条声明时,中间可以用分号【;】分隔。 最后一条声明可以没有分号,但是为了后续修改方便,一般也加上分号。为了使样式更加容易阅读,可以将每条声明单独成行。

\begin{lstlisting}[style=htmlcssjs, title=修改字体大小和颜色]
<!DOCTYPE html>
<html lang="en">
<head>
    <meta charset="UTF-8">
    <title>修改字体大小和颜色</title>
    <style type="text/css">
        p {
            font-size: 20px;
            color: red;
        }
    </style>
</head>
<body>
    <p>修改字体大小和颜色</p>
</body>
</html>
\end{lstlisting}

\newpage

\section{既然那么好,那就引入CSS吧——内联式CSS}

\subsection{内联式CSS}

内联式CSS样式,也称行间样式,就是把CSS代码直接写在现有的HTML标签中。注意CSS样式必须写在元素的开始标签里,不能在结束标签里。 \\

\begin{lstlisting}[style=htmlcssjs]
<开始标签 style="属性: 值;">文本</结束标签>
\end{lstlisting}

CSS样式必须写在style属性的双引号中,如果有多条CSS样式代码可以设置可以写在一起,中间用【;】隔开。

\begin{lstlisting}[style=htmlcssjs, title=内联式CSS]
<!DOCTYPE html>
<html lang="en">
<head>
    <meta charset="UTF-8">
    <title>内联式CSS</title>
</head>
<body>
    <p style="color: red; font-size: 20px;">内联式CSS</p>
</body>
</html>
\end{lstlisting}

\newpage

\section{换个地方吧,行内太挤了——嵌入式CSS}

\subsection{嵌入式CSS}

嵌入式CSS样式,也称页面级CSS样式,就是把CSS样式代码写在\lstinline|<style>|之间,嵌入式CSS样式一般放在\lstinline|<head>|内。

\begin{lstlisting}[style=htmlcssjs, title=嵌入式CSS]
<!DOCTYPE html>
<html lang="en">
<head>
    <meta charset="UTF-8">
    <title>嵌入式CSS</title>
    <style type="text/css">
        span {
            color: red;
        }
    </style>
</head>
<body>
    <p><span>乔恩</span>找到<span>艾洛莉</span></p>
</body>
</html>
\end{lstlisting}

\newpage

\section{还是把HTML和CSS分开吧——外部式CSS}

\subsection{外部式CSS}

外部式CSS样式,也称外联式CSS样式,就是把CSS样式代码写一个单独的外部文件中,这个CSS样式文件以.css为扩展名。在\lstinline|<head>|内使用\lstinline|<link>|将CSS外部样式文件链接到HTML文件内。 \\

\begin{lstlisting}[style=htmlcssjs]
<link href="CSS样式文件名" rel="stylesheet" type="text/css" />
\end{lstlisting}

CSS样式文件名以有意义的英文命名,\lstinline|<link>|中\lstinline|rel="stylesheet" type="text/css"|是固定写法,不需要修改。

\begin{lstlisting}[style=htmlcssjs, title=external\_css.html]
<!DOCTYPE html>
<html lang="en">
<head>
    <meta charset="UTF-8">
    <title>外部式CSS</title>
    <link type="text/css" rel="stylesheet" href="external_css.css">
</head>
<body>
    <div></div>
</body>
</html>
\end{lstlisting}

\begin{lstlisting}[style=htmlcssjs, title=external\_css.css]
div {
    width: 100px;
    height: 100px;
    background-color: blue;
}
\end{lstlisting}

\newpage

\section{总有个先来后到吧——三种链接方式的优先级}

\subsection{CSS引入方式优先级}

如果有一种情况:对于同一个元素同时使用了三种方法设置CSS样式,那么哪种方式真正有效呢? \\

三种CSS引入方式是有优先级的:内联式 > 嵌入式 > 外部式。但是嵌入式 > 外部式有一个前提,那就是嵌入式CSS样式的位置一定在外部式的后面。在实际开发中也会将\lstinline|<link>|写在\lstinline|<style>|的前面。 \\

总的来说,优先级遵循“就近原则”,离被设置元素越近优先级别越高。 \\

但是以上总结的优先级有一个前提,那就是内联式、嵌入式、外部式样式表中CSS样式是在相同权值的情况下。那权值是什么呢?

\begin{figure}[H]
    \centering
    \includegraphics[]{img/C5/5-5/1.png}
\end{figure}

\newpage