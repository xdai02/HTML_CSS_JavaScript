\chapter{语义化标签}

\section{开始我们的第一段对话吧}

\subsection{语义化(Semantic)}

学习HTML标签需要注意标签的用途和标签在浏览器中的默认样式。 \\

语义化,通俗的讲就是明白每个标签的用途,即在什么情况下使用此标签合理。比如,网页上文章的标题可以用标题标签、各个栏目的名称也可以使用标题标签、文章内容的段落就得放在段落标签中。 \\

语义化可以带来一些好处:

\begin{itemize}
    \item 更容易被搜索引擎收录
    \item 更容易让屏幕阅读器读出网页内容
\end{itemize}

\subsection{p标签}

如果想在网页上显示文章,就需要使用\lstinline|<p>|了,把文章的段落放到\lstinline|<p>|中。 \\

\begin{lstlisting}[style=htmlcssjs]
<p>段落文本</p>
\end{lstlisting}

注意一段文字一个\lstinline|<p>|标签,如在一篇文章中有三段文字,就要分别放到三个\lstinline|<p>|标签中。

\begin{lstlisting}[style=htmlcssjs, title=p标签]
<!DOCTYPE HTML>
<html lang="en">
<head>
    <meta charset="UTF-8">
    <title>p标签</title>
</head>
<body>
    <p>所有主流浏览器都支持p标签。</p>
    <p>p标签定义段落。</p>
    <p>p元素会自动在其前后创建一些空白。</p>
</body>
</html>
\end{lstlisting}

\lstinline|<p>|的默认样式,在段前段后都会有空白,如果不喜欢这个空白,可以用CSS样式来删除或改变它。

\subsection{span标签}

\lstinline|<span>|是没有语义的,它的作用就是为了设置单独的样式用的。 \\

\begin{lstlisting}[style=htmlcssjs]
<span>文本</span>
\end{lstlisting}

如果现在想把一段中某些字设置成蓝色,这种情况下就可以用到\lstinline|<span>|了。

\begin{lstlisting}[style=htmlcssjs, title=span标签]
<!DOCTYPE HTML>
<html lang="en">
<head>
    <meta charset="UTF-8">
    <title>span标签</title>
    <style type="text/css">
        span {
            color: blue;
        }
    </style>
</head>
<body>
    <p>
        <span>莫里亚蒂</span>有份包裹指明要交给<span>夏洛克</span>
    </p>
</body>
</html>
\end{lstlisting}

\newpage

\section{做个标题党——hx标签}

\subsection{hx标签}

文章的段落用\lstinline|<p>|,那么文章的标题可以使用标题标签。标题标签一共有6个,\lstinline|<h1>|、\lstinline|<h2>|、\lstinline|<h3>|、\lstinline|<h4>|、\lstinline|<h5>|、\lstinline|<h6>|分别为一级标题、二级标题、三级标题、四级标题、五级标题、六级标题,并且依据重要性递减,\lstinline|<h1>|是最高的等级。 \\

\begin{lstlisting}[style=htmlcssjs]
<h1>标题文本</h1>
<h2>标题文本</h2>
<h3>标题文本</h3>
<h4>标题文本</h4>
<h5>标题文本</h5>
<h6>标题文本</h6>
\end{lstlisting}

网页上的各个栏目标题也可使用标题标签。因为\lstinline|<h1>|在网页中比较重要,一般\lstinline|<h1>|被用在网站名称上。 \\

标题标签的样式都会加粗,\lstinline|<h1>|字号最大,\lstinline|<h2>|字号相对\lstinline|<h1>|要小,以此类推\lstinline|<h6>|的字号最小。

\newpage

\section{div标签}

\subsection{div标签}

网页制作过程过中,可以把一些独立的逻辑部分划分出来,放在一个\lstinline|<div>|中,\lstinline|<div>|的作用就相当于一个容器。 \\

逻辑部分是页面上相互关联的一组元素,如网页中的独立的栏目版块,就是一个典型的逻辑部分。如下图所示,图中用红色边框标出的部分就是一个逻辑部分,就可以使用\lstinline|<div>|作为容器。 \\

\begin{figure}[H]
	\centering
	\includegraphics[scale=0.5]{img/C2/2-3/1.png}
\end{figure}

\begin{lstlisting}[style=htmlcssjs, title=div标签]
<!DOCTYPE html>
<html lang="en">
<head>
    <meta charset="UTF-8">
    <title>div标签</title>
</head>
<body>
    <div>
    <h2>热门课程排行榜</h2>
        <ol>
            <li>前端开发面试心法</li>
            <li>零基础学习HTML</li>
            <li>JavaScript全攻略</li>
        </ol>
    </div>

    <div>
        <h2>最新课程排行</h2>
        <ol>
            <li>版本管理工具介绍—Git篇</li>
            <li>Canvas绘图详解</li>
            <li>QQ5.0侧滑菜单</li>
        </ol>
    </div>
</body>
</html>
\end{lstlisting}

\newpage

\section{写代码都不换行吗?——br标签}

\subsection{br标签}

在HTML代码中输入回车、空格都是没有作用的,在HTML中是忽略回车和空格的,输入再多的回车和空格也是现实不出来的。如果需要在HTML文本中输入回车换行,就必须使用\lstinline|<br/>|。在需要加回车换行的地方加入\lstinline|<br/>|,\lstinline|<br/>|的作用相当于Word文档中的回车。 \\

\lstinline|<br/>|是一个单标签,没有HTML内容的标签就是单标签。单标签只需要写一个开始标签,这样的标签有\lstinline|<br/>|、\lstinline|<hr/>|和\lstinline|<img/>|。

\begin{lstlisting}[style=htmlcssjs, title=br标签]
<!DOCTYPE html>
<html lang="en">
<head>
    <meta charset="UTF-8">
    <title>br标签</title>
</head>
<body>
    <h2>《望庐山瀑布》</h2>
    <p>唐·李白</p>
    <p>
        日照香炉生紫烟,<br/>
        遥看瀑布挂前川。<br/>
        飞流直下三千尺,<br/>
        疑是银河落九天。
    </p>
</body>
</html>
\end{lstlisting}

\newpage

\section{再加点空格呢?——特殊字符}

\subsection{特殊字符}

在HTML代码中输入空格、回车都是没有作用的,输出多个空格只会显示1个空格。如果需要输入空格,必须使用特殊字符\lstinline|&nbsp;|。

\begin{lstlisting}[style=htmlcssjs, title=特殊字符]
<!DOCTYPE html>
<html lang="en">
<head>
    <meta charset="UTF-8">
    <title>空格</title>
</head>
<body>
    <h2>新闻纵览</h2>
    来源:XXX&nbsp;&nbsp;&nbsp;&nbsp;&nbsp;作者:XXX
</body>
</html>
\end{lstlisting}

\newpage

\section{再来个水平分割线——hr标签}

\subsection{hr标签}

在信息展示时,有时会需要加一些用于分隔的横线,这样会使文章看起来整齐些。 \\

\lstinline|<hr/>|和\lstinline|<br/>|一样也是一个单标签,所以只有一个开始标签,没有结束标签。 \\

\lstinline|<hr/>|在浏览器中的默认样式线条比较粗,颜色为灰色。有些人觉得这种样式不美观,这些在样式可以通过CSS进行修改。

\begin{lstlisting}[style=htmlcssjs, title=hr标签]
<!DOCTYPE html>
<html lang="en">
<head>
    <meta charset="UTF-8">
    <title>hr标签</title>
</head>
<body>
    <p>深夜,一道黑影潜入贝克街221B。</p>
    <hr/>
    <p>坐在暗处的夏洛克已等候多时,正等着贝尔纳多自投罗网。</p>
</body>
</html>
\end{lstlisting}

\newpage